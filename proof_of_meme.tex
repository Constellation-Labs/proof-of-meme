\documentclass{article}
\usepackage[utf8]{inputenc}
\usepackage{listings}
\usepackage{float}
\usepackage{natbib}
\usepackage{graphicx}
\usepackage{amssymb}
\usepackage{amsmath}
\usepackage{mathtools}
\usepackage{listings}
\usepackage{color}
\usepackage{hyperref}

\definecolor{dkgreen}{rgb}{0,0.6,0}
\definecolor{gray}{rgb}{0.5,0.5,0.5}
\definecolor{mauve}{rgb}{0.58,0,0.82}

\title{MEME: A distributed consensus protocol for dynamically scalable networks}
\author{Wyatt Meldman-Floch}
\date{February 17 2018}
\setlength{\parskip}{1em}

\begin{document}
\maketitle

\begin{abstract}
We define a dynamically scalable, eventually consistent distributed ledger as a cellular automation governed by of the optimization of two objectives: maintaining an invariant measure of computational cost for rewriting history and optimally shuffling data over network such that total state is distance regular\footnote{distance regular \url{http://www.math.ucsd.edu/~fan/wp/green.pdf}} under toroidal search space\footnote{6.4.1 Toroidal Distance Equation \url{https://scholarworks.gsu.edu/cgi/viewcontent.cgi?article=1108&context=cs_diss}} (or can be queried within n-hops). These constraints are governed by a differential model\footnote{7 Networks with Communication Time-Delays \url{http://www.seas.ucla.edu/coopcontrol/papers/om03-tac.pdf}}, which we show is satisfied by a poincare protocol.

\end{abstract}
\setcounter{secnumdepth}{0}
\section{Introduction}
If we realize that our process of distributed consensus is hyperbolic\footnote{6.4.1 Toroidal Distance Equation \url{https://scholarworks.gsu.edu/cgi/viewcontent.cgi?article=1108&context=cs_diss}}, it follows that the corresponding poincare protocol is carried by toroidal homeomorphisms\footnote{2.2 \url{https://arxiv.org/pdf/1506.06945.pdf}}.

\section{Model Definition}
If we define our Merkel DAG as a topological fiber bundle\footnote{}, network stability is governed by the steady states of a differential equation. Because a poincare protocol is equivalently a poincare complex, the state space of the network's configuration complex is a differentiable manifold and can be governed by a differential model. We define a valid construction as a cellular automation who's cw-complex is poincare and who's manifold is a valid solution (state) to the governing model. 

\section{Planar cellular solution}
We explicitly define the functors and sheaves. By showing that this construction is a poincare protocol and that the protocol manifold (Planar cellular solution) adheres to the boundary conditions of the differential equation, we show equivalence. By equivalence, the steady states of the network correspond to modes of growth.

\section{Results and properties}
The result is that our planar cellular solution exhibits dynamic scaling. We call this dynamically scaling validation protocol, 'spectral consensus' as it corresponds to the graph spectra of the network at a given time. The intervals of these spectra follow from the fact that a 

\section{References}
Ansov flows\footnote{\url{https://www.math.upenn.edu/~ghrist/preprints/anosov.pdf}}

\bibliographystyle{plain}
\end{document}
